\section{Avoiding Programmer-induced Slow Paths}
\label{sec:advisories}

In this section, we show how compile-time performance advisories allow programmers to avoid performance pitfalls and how they can use reflection to expose internal state at run-time. %First, we will present the need for performance advisories and introduce the different types of advisories that exist today. A discussion about the motivation for introducing the MbReflection API will follow, along with examples to demonstrate its functionality.

\subsection{Performance Advisories}

%what are the performance advisories and what is their purpose
% 2 problems:
%	- programmer not aware of possible optimizations in the code
%	- programmer not aware of limitations of optimizer
% Silent failure - using performance advisories this is avoided
% Shown in form of compile-time warnings
% Improve code performance up to 5 times
A fundamentally difficult problem with using miniboxing is determining when the code has reverted to using boxed primitives. Having miniboxed code use boxed primitives is most commonly caused by interacting with erased generics and specialization, but can also be caused by technical or design limitations. While all these cases lead to correct and semantically-equivalent code, they do produce noticeable slowdowns. % These limitations can occur at different levels, from well-understood design decisions that sacrifice certain code patterns, to purely technical limitations where the problem is simply not worth the effort to solve.

When boundary conditions or limitations occur in a typical opaque transformation, the optimization process silently fails, and doesn't give the programmer any useful information about what happened. Instead, the miniboxing transformation provides the programmer with a detailed report of the error followed by the common workaround or links to in-depth discussions on the topic. Using this feature, novice programmers can fine-tune their programs without a thorough understanding of the transformation. %This is exactly the function of performance advisories in the Miniboxing plugin. They aim to provide the user with useful compile-time warnings that explain why code could not be fully optimized by specializing generics and detail how the code can be improved in order to achieve the best performance.

% Besides limitations of the optimizer, it is also possible for the code to be written in a manner that does not fully utilize compiler's optimization potential. Performance advisories of Minibox plugin will assist the programmer in this situation by exposing code where the plugin wasn't able to correctly specialize generics, pointing out exactly what the problem was and suggesting ways to fix it. Introduced performance advisories play an important role by helping improve the performance of a program significantly. Benchmarks show that applying the compilers suggested changes can improve overall performance by up to 4x.

There are as many as 10 different performance advisories implemented in the miniboxing plugin, but in order to focus on the concept, we will only look at the two most common advisories, both caused by the interaction with erased generics. To show exactly how the slowdowns occur, we can take the following piece of code:

\begin{lstlisting-nobreak}
 def foo[`@miniboxed` T](t: T): T = bar(t)
 def bar[`@miniboxed` T](t: T): T = baz(t)
 def baz[`@miniboxed` T](t: T): T = t
\end{lstlisting-nobreak}

The code is transformed to:

\begin{lstlisting-nobreak}
 def foo(t: Object): Object = bar(t)
 def bar(t: Object): Object = baz(t)
 def baz(t: Object): Object = t
 def foo`_J`(..., t: long): long = bar`_J`(..., t)
 def bar`_J`(..., t: long): long = baz`_J`(..., t)
 def baz`_J`(..., t: long): long = t
\end{lstlisting-nobreak}

The translation shows that once execution entered the miniboxed path, by calling |foo_J|, it goes through without any boxing, only passing the value in the encoded (miniboxed) representation. However, let us think of what happens if the |@miniboxed| annotation is removed from method |bar|:

\begin{lstlisting-nobreak}
 def foo[`@miniboxed` T](t: T): T = bar(t)
 def bar[T](t: T): T = baz(t)
 def baz[`@miniboxed` T](t: T): T = t
\end{lstlisting-nobreak}

This produces the following code:

\begin{lstlisting-nobreak}
 def foo(t: Object): Object = bar(t)
 def bar(t: Object): Object = baz(t)
 def baz(t: Object): Object = t
 def foo_J(..., t: long): long = `box2minibox(bar(minibox2box(t)))` // boxing :(
 def baz_J(..., t: long): long = t
\end{lstlisting-nobreak}

There are two problems that occur here:

\begin{compactitem}
 \item Method |foo| does not have a miniboxed version of |bar| to call;
 \item Once the erased |bar| method is called, the call stack is generic, despite the fact that there is a miniboxed version of |baz| (but it is not valid to call it, as the argument may not be a primitive).
\end{compactitem}

These two problems correspond exactly to the two types of performance advisories: forward and backward.

\subsubsection{Forward advisories.} The first advisory (compiler warning) received by the programmer is also called a forward warning:

\begin{lstlisting-nobreak-nolang}
test.scala:7: warning: The method bar would benefit from miniboxing type parameter T, since it is instantiated by miniboxed type parameter T of method foo:

       def foo[@miniboxed T](t: T): T = bar(t)
                                        ^
\end{lstlisting-nobreak-nolang}

The name forward comes from the fact that this advisory pushes the miniboxed representation from caller to callee each time the arguments need to be boxed before being passed.

\subsubsection{Backward advisories.} The miniboxing annotation is also propagated from callee to caller:

\begin{lstlisting-nobreak-nolang}
test.scala:8: warning: The following code could benefit from miniboxing specialization if the type parameter T of method bar would be marked as "@miniboxed T" (it would be used to instantiate miniboxed type parameter T of method baz):

        def bar[T](t: T): T = baz(t)
                              ^
\end{lstlisting-nobreak-nolang}

With these warnings, even a novice programmer, not familiar to either the language or the miniboxing transformation is still capable of restoring the optimality of the code.

\subsubsection{Implementation.} Although these warnings may seem very difficult to generate, they are actually less than 5 lines of code. Advisories are generated during the method rewiring decision, which decides which method to call (the miniboxed or the generic one). To do so, it looks at the type parameters and the type arguments in parallel:

\begin{lstlisting-nobreak}
 (tpars zip targs).flatMap {
   case (tpar, PrimitiveType) =>
     if (!tpar.isMiniboxed)
       // non-miniboxed type param instantiated by
       // primitive type -- forward warning
     (tpar, Miniboxed)
   case (tpar, Reference(tpar2)) =>
     case tpar.isMiniboxed && tpar2.isMiniboxed =>
       (tpar, Miniboxed)
     case !tpar.isMiniboxed && !tpar2.isMiniboxed =>
       (tpar, Erased)
     case tpar.isMiniboxed && !tpar2.isMiniboxed =>
       // non-miniboxed type param used to instantiate a
       // miniboxed type parameter -- backward warning
       (tpar, Erased)
     case !tpar.isMiniboxed && tpar2.isMiniboxed =>
       // miniboxed type param instantiated by erased
       // type parameter -- forward warning
       (tpar, Erased)
   ...
 }
\end{lstlisting-nobreak}

For example, given the following code:

\begin{lstlisting-nobreak}
 val five = foo[Int](5)
\end{lstlisting-nobreak}

The question asked is whether to transform the call to |foo| into a call to |foo_J|. To do this, we have to decide whether the type parameter of method |foo|, namely |T|, is instantiated by a primitive type or another miniboxed type parameter. In this case, since |Int| is a primitive type, the decision takes the first case, pairing type parameter |T| with the |Miniboxed| tag (and without warning, since |T| is miniboxed). Going back to the decision, the answer is positive, since the type parameter |T| of method |foo| is indeed instantiated by a primitive type:

\begin{lstlisting-nobreak}
 val five: int = minibox2int(foo`_J`(int2minibox(5)))
\end{lstlisting-nobreak}

\subsubsection{Suppressing warnings.} In certain scenarios, programmers are aware of their sub-optimal erased generic code but, due to compatibility requirements with other JVM programs or due to technical limitations, they chose not to change it. In these situations, they need to suppress the warnings, since instead of improving visibility, they prevent a clear picture of the sub-optimalities in the program. However, a coarse-grained approach such as turning off all warnings is not desirable either, since it hides other potentially valuable pieces of information. For this scenario, the miniboxing transformation provides the |@generic| annotation, which can suppress both forward and backward warnings:

\begin{lstlisting-nobreak}
 scala> def zoo[`@miniboxed` T](t: T) = t
 defined method zoo

 scala> zoo[Any `@generic`](3) // no forward warning
 res1: Any = 3

 scala> def boo[`@generic` T](t: T) = t
 defined method boo

 scala> boo[Int](3)                   // no backward warning
 res2: Int = 3
\end{lstlisting-nobreak}


\subsubsection{Libraries.} In other cases boxing is caused by the interaction with erased generics from libraries. In this case, the default decision is not to warn, unless the programmer specifically sets the |-P:minibox:warn-all| compiler flag:

\begin{lstlisting-nobreak}
 scala> 3 :: Nil
 <console>:8: warning: The method List.:: would benefit from miniboxing type parameter B, since it is instantiated by a primitive type:

               3 :: Nil
                 ^
 res0: List[Int] = List(3)
\end{lstlisting-nobreak}

As we will see in the benchmarking section (\S\ref{sec:bench}), the performance advisories allow programmers who are not familiar with the transformation to make the changes an expert in the area would do.

\subsection{Data Representation Reflection}

% motivation - define different run time behavior based on class type argument
%Performance advisories provide the programmer with compile-time warnings when code is sub-optimal, and as mentioned above, there are mechanisms to suppress them. This leads to cases where but still enforce strictness and define a separate run-time behavior for the class if its type argument is a primitive or reference type

This section will show how the miniboxing transformation offers a window into what code is running and how it was transformed. Consider the following two examples:

\begin{compactitem}
\item Having shut off some of the performance advisories, a programmer may still want to guarantee a class is only instantiated in its specialized form using reflection;
\item When designing a container, the actions may depend on the types of elements stored: In the immutable |RRB-Vector| \cite{nicolas-thesis} implementation, when elements from the front or back are removed, underlying arrays that store elements can be reused. Here, only starting and stopping indices of the resultant immutable vector are updated. However, if |Vector| is used to store references, reusing previous arrays can result in memory leaks (since the arrays keep a reference to the deleted object for the old vector, but create a reference path between the new vector and the deleted element). In this case, the ideal solution would be to change the behavior of the |Vector| class based on the type argument and only reuse the arrays for primitive types.
\end{compactitem}

% what is mbreflection
The |MbReflection| object allows the programmer to check the internal state of the plugin by reflecting on the type parameters of miniboxed classes and methods. For example, given a method with a miniboxed type parameter |T|, reflection can determine, at run-time, whether |T| is miniboxed, its instantiation, and the type used to store the value:

\begin{lstlisting-nobreak}
 scala> import MiniboxingReflection._
 import MiniboxingReflection._

 scala> def bip[@miniboxed T] = s"bip[T = ${`reifiedType[T]`}, miniboxed into a ${`storageType[T]`}]"
 defined method bip

 scala> bip[Int]
 res4: String = bip[T = Int, miniboxed into a Long]
\end{lstlisting-nobreak}

The method |reifiedType[T]| indicates the type of type parameter instantiation while the |storageType[T]| returns the type used for encoding data. It is also possible to call the method from an erased context, in which case it will report the fact that the type parameter is a reference to a boxed object:

\begin{lstlisting-nobreak}
 scala> def bop[T] = bip[T]
 // backward warning...
 defined method bop

 scala> bop[Int]
 // forward warning...
 res5: String = bip[T = Reference, miniboxed into a Reference]
\end{lstlisting-nobreak}

From an implementation point of view, the reflection methods are simply bridged after duplication to either constant values (in the case of the |_L| variants of the classes and methods) or to the type byte mentioned in \S\ref{sec:minibox}, for the |_J| variants. This is a simple mechanism which offers programmers yet another way to avoid slow paths in their programs.  Other reflection features include the partial evaluation of reflection method calls when the result is statically known and optimized conversion methods that allow converting values to and from the miniboxed representation efficiently. Instead of presenting these features in detail, we will now focus on the inter-operation between miniboxing and specialized library classes, which exposes a completely different type of slow path.